\section{Requisiti Software}
\label{sec:requisiti_software}
In questa sezione vengono elencati i \glossario{requisiti software} individuati dal gruppo per la realizzazione del progetto ArtificialQI.
I requisiti sono stati definiti tramite l'analisi dei \glossario{requisiti utente} tramite i diagrammi \glossario{use case}.
I \glossario{requisiti software} si dividono in:
\begin{enumerate}
    \label{en:req_type}
    \item Requisiti funzionali: definiscono le funzionalità attese dal sistema.
    \item Requisiti non funzionali: definiscono gli attributi attesi del sistema nel suo insieme.
    Si dividono in:
    \begin{enumerate}
        \item Requisiti di qualità: impongono dei vincoli sulla qualità del prodotto.
        \item Requisiti di vincolo: impongono dei vincoli sullo svolgimento del progetto.
    \end{enumerate}
\end{enumerate}
Ogni \glossario{requisito software} viene descritto indicando le seguenti informazioni:
\begin{enumerate}
    \item Identificativo che segue la sintassi:
    \begin{lstlisting}
        R<T><O/F>-<N>
    \end{lstlisting}
    Dove:
    \begin{enumerate}
        \item \lstinline{T} indica il tipo di requisito software che può variare tra \lstinline{F}(funzionali), \lstinline{Q}(di qualità) e \lstinline{V}(di vincolo).
        \item \lstinline{O} indica che il requisito è obbligatorio mentre \lstinline{F} indica che il requisito è facoltativo.
        \item \lstinline{N} è un numero crescente rispetto al tipo di requisito che parte dal valore 1.
    \end{enumerate}
    \item Descrizione testuale.
    \item Specifica ovvero la lista di sotto requisiti che approfondiscono il \glossario{requisito software} principale.
    Ogni sotto requisito è identificato usando l'identificativo del requisito padre a cui viene aggiunto in coda un secondo numero crescente che parte da 1 preceduto da un
    \item Dipendenze con altri \glossario{requisito software} facoltativi.
    \item Una lista dei requisiti utente che hanno dato origine al \glossario{requisito software}.
\end{enumerate}

\subsection{Requisiti funzionali}
\label{subsec:requisiti_funzionali}
I requisiti funzionali sono raggruppati usando una divisione ad alto livello del sistema in due componenti, ovvero 
\begin{enumerate}
    \item Logica di presentazione: si occupa della presentazione delle informazioni agli utenti.
    \item Logica di business: si occupa dell'elaborazione delle informazioni mostrate agli utenti e ottenute dagli utenti.
\end{enumerate}

\subsubsection{Logica di presentazione}

\paragraph{Obbligatori}

\begin{freq}
    {RFO-1}
    {Il sistema deve permettere la visualizzazione di un messaggio che indica lo stato delle tab aperte}
    \label{rf:RFO-1}%

    \subreq{RFO-1.1}{\hyperref[uc:UC-4]{UC-4}}{Il sistema deve visualizzare un messaggio che indica l'assenza di una tab di Instagram aperta}
    \subreq{RFO-1.2}{\hyperref[uc:UC-5]{UC-5}}{Il sistema deve visualizzare un messaggio che indica l'assenza del login nella tab di Instagram aperta}
\end{freq}

\begin{freq}
    {RFO-2}
    {Il sistema deve poter visualizzare gli elementi necessari per permettere la raccolta dei dati}
    \label{rf:RF2-3}%

    \subreq{RFO-2.1}{\hyperref[uc:UC-8]{UC-8}}{Il sistema deve visualizzare un pulsante che permetta di iniziare la raccolta dati}
\end{freq}

\begin{freq}
    {RFO-3}
    {Il sistema deve poter visualizzare gli elementi necessari per notificare l'esito della raccolta dati}
    \label{rf:RFO-3}%

    \subreq{RFO-3.1}{\hyperref[uc:UC-8]{UC-8}}{Il sistema deve visualizzare un messaggio che informa l'utente che la raccolta dei dati è in esecuzione}
    \subreq{RFO-3.2}{\hyperref[uc:UC-8]{UC-8}}{Il sistema deve visualizzare un messaggio che indica il successo della raccolta dati}
    \subreq{RFO-3.3}{\hyperref[uc:UC-8]{UC-8}}{Il sistema deve visualizzare un link all pagina dell'ultimo post di cui sono stati raccolti i dati}
    \subreq{RFO-3.4}{\hyperref[uc:UC-8]{UC-8}}{Il sistema deve visualizzare un messaggio che indica il numero di post di cui sono stati raccolti i dati e il numero di post totale del profilo}
    \subreq{RFO-3.5}{\hyperref[uc:UC-9]{UC-9}}{Se il caricamento della pagina di un post durante la raccolta non viene completato nei tempi previsti, il sistema deve visualizzare un messaggio che informa l'utente dell'errore}
\end{freq}

\begin{freq}
    {RFO-4}
    {Il sistema deve poter visualizzare gli elementi necessari per permettere la pubblicazione delle risposte ai nuovi commenti rilevati}
    \label{rf:RFO-4}%

    \subreq{RFO-4.1}{\hyperref[uc:UC-10]{UC-10}}{Il sistema deve visualizzare un pulsante che permetta di iniziare la pubblicazione delle risposte ai nuovi commenti}
\end{freq}

\begin{freq}
    {RFO-5}
    {Il sistema deve poter visualizzare gli elementi necessari per notificare l'esito della pubblicazione delle risposte ai nuovi commenti}
    \label{rf:RFO-5}%

    \subreq{RFO-5.1}{\hyperref[uc:UC-10]{UC-10}}{Il sistema deve visualizzare un messaggio che informa l'utente che la pubblicazione delle risposte è in esecuzione}
    \subreq{RFO-5.2}{\hyperref[uc:UC-10]{UC-10}}{Il sistema deve visualizzare un messaggio che indica il successo della pubblicazione delle risposte ai commenti}
    \subreq{RFO-5.3}{\hyperref[uc:UC-9]{UC-9}}{Se il caricamento della pagina di un post durante la pubblicazione non viene completato nei tempi previsti, il sistema deve visualizzare un messaggio che informa l'utente dell'errore}
    \subreq{RFO-5.4}{\hyperref[uc:UC-11]{UC-11}}{Se si verifica un errore durante il recupero delle risposte ai commenti da pubblicare, il sistema deve visualizzare un messaggio che informa l'utente dell'errore}
\end{freq}

\begin{freq}
    {RFO-6}
    {Il sistema deve poter visualizzare gli elementi necessari per permettere il download dei dati raccolti in un file JSON}
    \label{rf:RFO-6}%

    \subreq{RFO-6.1}{\hyperref[uc:UC-12]{UC-12}}{Il sistema deve visualizzare un pulsante che permetta di iniziare il download del file JSON con i nuovi dati raccolti}
    \subreq{RFO-6.2}{\hyperref[uc:UC-12]{UC-12}}{Il sistema deve visualizzare una finestra di dialogo che permetta di selezionare la cartella dove salvare il file}

\end{freq}

\begin{freq}
    {RFO-7}
    {Il sistema deve visualizzare gli elementi necessari per notificare l'esito del download del file}
    \label{rf:RFO-7}%

    \subreq{RFO-7.1}{\hyperref[uc:UC-12]{UC-12}}{Il sistema deve visualizzare un messaggio che indica il successo del download del file JSON}
    \subreq{RFO-7.2}{\hyperref[uc:UC-13]{UC-13}}{Se il download del file viene interrotto, il sistema deve visualizzare un messaggio che informa l'utente dell'interruzione}
\end{freq}

\paragraph{Facoltativi}

\begin{freq}
    {RFF-1}
    {Il sistema deve permettere la visualizzazione di un messaggio che indica il risultato dell'analisi del layout di Instagram}
    \label{rf:RFF-1}%

    \subreq{RFF-1.1}{\hyperref[uc:UC-1]{UC-1}}{Il sistema deve visualizzare un messaggio che informa l'utente che l'analisi del layout di Instagram è in esecuzione}
    \subreq{RFF-1.2}{\hyperref[uc:UC-1]{UC-1}}{Se l'analisi del layout di Instagram ha esito positivo, il sistema deve mostrare un messaggio che ne conferma il successo}
    \subreq{RFF-1.3}{\hyperref[uc:UC-2]{UC-2}}{Se l'analisi del layout di Instagram ha esito negativo, il sistema deve visualizzare un messaggio che indica l'impossibilità di effettuare una raccolta dati}
    \subreq{RFO-1.4}{\hyperref[uc:UC-2]{UC-2}}{Se l'analisi del layout di Instagram ha esito negativo, il sistema deve visualizzare un messaggio che indica ogni elemento del layout Instagram non trovato}
\end{freq}

\begin{freq}
    {RFF-2}
    {Il sistema deve poter visualizzare gli elementi necessari per notificare l'esito del calcolo delle differenze tra i dati raccolti}
    \label{rf:RFF-2}%

    \subreq{RFO-2.1}{\hyperref[uc:UC-6]{UC-6}}{Se vengono rilevate differenze tra i dati dell'ultima raccolta e quelli della raccolta precedente, il sistema deve visualizzare un messaggio che informa l'utente della presenza di variazioni nei nuovi dati raccolti}
    \subreq{RFO-2.2}{\hyperref[uc:UC-6]{UC-6}}{Se non vengono rilevate differenze tra i dati dell'ultima raccolta e quelli della raccolta precedente, il sistema deve visualizzare un messaggio che informa l'utente che i nuovi dati raccolti coincidono con quelli precedenti}
    \subreq{RFO-2.3}{\hyperref[uc:UC-7]{UC-7}}{Se si verifica un errore durante l'ottenimento dei dati delle ultime due raccolte, il sistema deve visualizzare un messaggio che informa l'utente dell'errore}
\end{freq}

\subsubsection{Logica di business}

\paragraph{Obbligatori}

\begin{freq}
    {RFO-8}
    {Il sistema deve poter gestire la persistenza dei dati raccolti}
    \label{rf:RFO-8}%

    \subreq{RFO-8.1}{\hyperref[uc:UC-8]{UC-8}}{Il sistema deve poter salvare in modo persistente i dati ottenuti durante una raccolta}
    \subreq{RFO-8.2}{\hyperref[uc:UC-6]{UC-6}, \hyperref[uc:UC-8]{UC-8}, \hyperref[uc:UC-12]{UC-12}}{Il sistema deve poter ottenere i dati salvati durante l'ultima raccolta}
\end{freq}

\begin{freq}
    {RFO-9}
    {Il sistema deve poter gestire le tab aperte}
    \label{rf:RFO-9}%

    \subreq{RFO-9.1}{\hyperref[uc:UC-3]{UC-3}}{Il sistema deve poter ottenere una lista delle tab aperte nella finestra del Browser}
    \subreq{RFO-9.2}{\hyperref[uc:UC-4]{UC-4}}{Il sistema deve poter ottenere l'URL di ogni tab aperta}
    \subreq{RFO-9.3}{\hyperref[uc:UC-4]{UC-4}}{Il sistema deve poter verificare se l'URL di una tab aperta corrisponde a un dominio di Instagram}
    \subreq{RFO-9.4}{\hyperref[uc:UC-5]{UC-5}}{Il sistema deve essere in grado di acquisire i cookie da una tab aperta}
    \subreq{RFO-9.5}{\hyperref[uc:UC-5]{UC-5}}{Il sistema deve controllare che siano presenti i cookie di autenticazione di Instagram}
\end{freq}

\begin{freq}
    {RFO-10}
    {Il sistema deve poter navigare tra le pagine di Instagram}
    \label{rf:RFO-10}%

    \subreq{RFO-10.1}{\hyperref[uc:UC-8]{UC-8}, \hyperref[uc:UC-10]{UC-10}}{Il sistema deve poter ottenere il pulsante nel layout Instagram per accedere alla pagina del profilo personale}
    \subreq{RFO-10.2}{\hyperref[uc:UC-8]{UC-8}}{Il sistema deve poter ottenere il numero di post totali pubblicati nel profilo personale}
    \subreq{RFO-10.3}{\hyperref[uc:UC-8]{UC-8}, \hyperref[uc:UC-10]{UC-10}}{Il sistema deve poter ottenere il link per caricare l'ultimo post pubblicato nel profilo personale}
    \subreq{RFO-10.4}{\hyperref[uc:UC-8]{UC-8}, \hyperref[uc:UC-10]{UC-10}}{Il sistema deve poter ottenere il pulsante nel layout Instagram per navigare al post successivo}
    \subreq{RFO-10.5}{\hyperref[uc:UC-8]{UC-8}, \hyperref[uc:UC-10]{UC-10}, \hyperref[uc:UC-9]{UC-9}}{Il sistema deve poter determinare il cambiamento dell'URL della pagina}
    \subreq{RFO-10.6}{\hyperref[uc:UC-8]{UC-8}, \hyperref[uc:UC-10]{UC-10}}{Il sistema deve poter determinare cambiamenti avvenuti al DOM della pagina}
\end{freq}

\begin{freq}
    {RFO-11}
    {Il sistema deve poter ottenere i dati di un post di Instagram}
    \label{rf:RFO-11}%

    \subreq{RFO-11.1}{\hyperref[uc:UC-8]{UC-8}}{Il sistema deve poter ottenere il numero del post}
    \subreq{RFO-11.2}{\hyperref[uc:UC-8]{UC-8}}{Il sistema deve poter ottenere la caption del post}
    \subreq{RFO-11.3}{\hyperref[uc:UC-8]{UC-8}}{Il sistema deve poter ottenere il numero di likes al post}
    \subreq{RFO-11.4}{\hyperref[uc:UC-8]{UC-8}, \hyperref[uc:UC-10]{UC-10}}{Il sistema deve poter ottenere l'URL della pagina del post}
    \subreq{RFO-11.5}{\hyperref[uc:UC-8]{UC-8}}{Il sistema deve poter ottenere il valore dell'attributo "src" nel caso in cui il post sia un'immagine}
    \subreq{RFO-11.6}{\hyperref[uc:UC-8]{UC-8}, \hyperref[uc:UC-10]{UC-10}}{Il sistema deve poter ottenere testo e autore di tutti i commenti al post che non contengono solo immagini}
    \subreq{RFO-11.7}{\hyperref[uc:UC-8]{UC-8}, \hyperref[uc:UC-10]{UC-10}}{Il sistema deve poter ottenere il pulsante per caricare i nuovi commenti}

\end{freq}

\begin{freq}
    {RFO-12}
    {Il sistema deve poter pubblicare una risposta ad un commento su un post di Instagram}
    \label{rf:RFO-12}%

    \subreq{RFO-12.1}{\hyperref[uc:UC-10]{UC-10}}{Il sistema deve poter identificare il commento corrispondente dato il testo, l'autore e l'URL del post in cui è pubblicato}
    \subreq{RFO-12.2}{\hyperref[uc:UC-10]{UC-10}}{Il sistema deve poter ottenere il pulsante per rispondere ad un commento}
    \subreq{RFO-12.3}{\hyperref[uc:UC-10]{UC-10}}{Il sistema deve poter ottenere l'area di testo in cui inserire il testo della risposta}
    \subreq{RFO-12.4}{\hyperref[uc:UC-10]{UC-10}}{Il sistema deve poter inserire correttamente il testo della risposta nell'area di testo}
    \subreq{RFO-12.5}{\hyperref[uc:UC-10]{UC-10}}{Il sistema deve poter ottenere il pulsante per pubblicare la risposta al commento}
    \subreq{RFO-12.6}{\hyperref[uc:UC-10]{UC-10}, \hyperref[uc:UC-11]{UC-11}}{Il sistema deve poter ottenere le risposte ai commenti da pubblicare}

\end{freq}

\begin{freq}
    {RFO-13}
    {Il sistema deve poter gestire il download di un file JSON}
    \label{rf:RFO-13}%

    \subreq{RFO-13.1}{\hyperref[uc:UC-12]{UC-12}}{Il sistema deve poter effettuare il download di un file nel file system locale}
    \subreq{RFO-13.2}{\hyperref[uc:UC-13]{UC-13}}{Il sistema deve poter determinare lo stato del download in corso}
\end{freq}

\paragraph{Facoltativi}

\begin{freq}
    {RFF-3}
    {Il sistema deve poter analizzare il layout di Instagram}
    \label{rf:RFF-3}%

    \subreq{RFF-3.1}{\hyperref[uc:UC-1]{UC-1}}{Il sistema deve poter aprire una tab di Instagram temporanea di un account di test}
    \subreq{RFF-3.2}{\hyperref[uc:UC-1]{UC-1}, \hyperref[uc:UC-2]{UC-2}}{Il sistema deve essere in grado di verificare che ciascun selettore di un elemento della pagina di Instagram individui correttamente l'elemento corrispondente nella pagina dell'account di test}
\end{freq}

\begin{freq}
    {RFF-4}
    {Il sistema deve gestire le differenze tra i dati raccolti nelle esecuzioni consecutive}
    \label{rf:RFF-4}%

    \subreq{RFF-4.1}{\hyperref[uc:UC-6]{UC-6}}{Il sistema deve poter calcolare le differenze tra due set di dati raccolti}
    \subreq{RFF-4.2}{\hyperref[uc:UC-6]{UC-6}}{Il sistema deve rappresentare le differenze calcolate in un formato JSON strutturato}
\end{freq}

\subsection{Requisiti non funzionali}
\label{subsec:requisiti_non_funzionali}

\subsubsection{Requisiti di vincolo}

\paragraph{Obbligatori}
\label{rnf:RVO-1}%
\nfreq{RVO-1}{Il sistema deve essere prodotto come estensione Chrome}{Discussione con tutor aziendale}
\label{rnf:RVO-2}%
\nfreq{RVO-2}{Il sistema deve essere supportato dall'ultima major version del browser Chrome}{Discussione con tutor aziendale}
\label{rnf:RVO-3}%
\nfreq{RVO-3}{Il progetto deve essere reso disponibile tramite GitHub}{Discussione con tutor aziendale}

\newpage
\subsubsection{Requisiti di qualità}

\paragraph{Obbligatori}

\label{rnf:RQO-1}%
\nfreq{RQO-1}{L'utente deve poter aprire e chiudere l'estensione in modo rapido e intuitivo}{Discussione con tutor aziendale}
\label{rnf:RQO-2}%
\nfreq{RQO-2}{I selettori utilizzati per il parsing della pagina web devono essere indipendenti dalla lingua e ben isolati nel codice}{Discussione con tutor aziendale}

\paragraph{Facoltativi}
\label{rnf:RQF-1}%
\nfreq{RQF-1}{L'estensione deve rimanere attiva su tutte le tab}{Discussione con tutor aziendale}
\label{rnf:RQF-2}%
\nfreq{RQF-2}{L'estensione deve rimanere attiva e garantire il suo corretto funzionamento anche quando l'utente interagisce con altre schede o pagine all'interno della stessa finestra del browser}{Discussione con tutor aziendale}


   

