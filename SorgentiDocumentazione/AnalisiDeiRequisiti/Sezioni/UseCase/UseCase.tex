\section{Use case}
\label{sec:use_case}
Gli \glossario{use case} sono uno strumento grafico che permette di analizzare e approfondire i \glossario{requisiti utente} per arrivare a definire i \glossario{requisiti software} del sistema.

\input{Sezioni/UseCase/Sottosezioni/VisualizzazioneDatasetCaricato.tex}

\input{Sezioni/UseCase/Sottosezioni/ModificaDatasetVisualizzato.tex}

\input{Sezioni/UseCase/Sottosezioni/SalvataggioDataset.tex}

\input{Sezioni/UseCase/Sottosezioni/VisualizzazioneDatasetSalvati.tex}

\input{Sezioni/UseCase/Sottosezioni/ModificaDatasetSalvato.tex}

\input{Sezioni/UseCase/Sottosezioni/CambioDatasetCaricato.tex}

\input{Sezioni/UseCase/Sottosezioni/EsecuzioneTest.tex}

\input{Sezioni/UseCase/Sottosezioni/InterazioneDiagrammaDispersione.tex}

\input{Sezioni/UseCase/Sottosezioni/SalvataggioTest.tex}

\input{Sezioni/UseCase/Sottosezioni/VisualizzazioneTestSalvati.tex}

\input{Sezioni/UseCase/Sottosezioni/ModificaTestSalvato.tex}

\input{Sezioni/UseCase/Sottosezioni/ConfrontoTest.tex}

\input{Sezioni/UseCase/Sottosezioni/AggiungiLLM.tex}

\input{Sezioni/UseCase/Sottosezioni/ModificaLLM.tex}

\input{Sezioni/UseCase/Sottosezioni/VisualizzazioneLLMSalvati.tex}